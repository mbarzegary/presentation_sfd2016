\section{Play Your Notes: Embedded Systems}

\begin{frame}
  \frametitle{Play Your Notes: Embedded Systems}
  
  \begin{itemize}
      \item What is an Embedded System?
      \item What makes Embedded Systems different?
      \item Embedded Systems Complexity
      \item To Be or Not to Be, That is the Question
  \end{itemize}
  
    \centering
  \includegraphics[width=3cm]{tux_bass}

\end{frame}


\begin{frame}{What is an Embedded System?}

    \begin{block}{}
        An embedded system is a computer system with a \textbf{dedicated function} within a larger mechanical or electrical system, often with real-time computing constraints. It is \textbf{embedded} as part of a complete device often including hardware and mechanical parts. Embedded systems control many devices in common use today
    \end{block}
    
        \centering
  \includegraphics[width=2.5cm]{tux_worker}
    
    
\end{frame}


\begin{frame}
  \frametitle{What makes Embedded Systems different?}
  
  \begin{itemize}
      \item Real-time operation
        \item Size
        \item Cost
        \item Time
        \item Reliability
        \item Safety
        \item Energy
        \item Security
  \end{itemize}
\end{frame}

\begin{frame}
  \frametitle{Embedded Systems Complexity}
  
  \begin{block}{Moore's Law (Second edition)}
    The observation made in 1975 by Gordon Moore, co-founder of Intel, that the number of transistors per square inch on integrated circuits had doubled every double year since the integrated circuit was invented.
  \end{block}
  
      \centering
  \includegraphics[width=2.5cm]{tux_moore}
  
\end{frame}


\begin{frame}{To Be or Not to Be, That is the Question}
    
    \begin{block}{}
    So, why does your TV run Linux? At frst glance, the function of a TV is simple: it has to display a stream of video on a screen. Why is a complex Unix-like operating system like Linux necessary?
    \end{block}
    
    \\
    \\
    
        \centering
  \includegraphics[width=3cm]{tux_inside}
    
    
\end{frame}

